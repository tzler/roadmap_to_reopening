% PREAMBLE
% \documentclass[twoside,titlepage]{article}
\documentclass[12pt, titlepage, letterpaper]{article}
\usepackage[utf8]{inputenc}
\usepackage[margin=1in]{geometry}
% \geometry{legalpaper}%, margin=1in}
\usepackage{graphicx}
\usepackage[numbers, super]{natbib}
%\usepackage{authblk}
\usepackage[dvipsnames]{xcolor}
%\usepackage{amsmath}
\usepackage[citecolor=blue,colorlinks=true]{hyperref}
%\usepackage{anyfontsize}
\usepackage[all]{hypcap}
\usepackage{epigraph}
\usepackage{enumitem}
%\usepackage{helvet}
% \setlength{\parindent}{0em}
\usepackage{soul}
\usepackage{ragged2e} %, sectsty}
\usepackage[skip=1em]{parskip}
\setitemize{noitemsep}
\usepackage{pdfpages} 
% \allsectionsfont{\sffamily}
%% item type
\renewcommand{\labelitemii}{$\circ$}
\frenchspacing

\definecolor{dark-red-custom}{HTML}{820000}
\hypersetup{
    urlcolor=dark-red-custom,
    linkcolor=BlueViolet,
}
% \definecolor{dark-blue-custom}{HTML}{820000}


\title{\vspace{-2em}\textbf{A Roadmap to Reopening \\ with Justice and Equity}}
\author{}
\date{%\begin{center}  

\vspace{-4em}
\normalsize
\textit{The ``Roadmap to Reopening with Justice and Equity” is a coalition-led effort \\spearheaded by the following organizations:}

Abolish Stanford
\\Black Graduate Student Association (BGSA)
\\Graduate Student Council (GSC)
\\Sexual Violence Free Stanford (SVFS)
\\Stanford Basic Needs Coalition (BNC)
\\Stanford Neighbor Accountability Coalition (SNAC)
\\Stanford Solidarity Network (SSN)
\\Students for Workers’ Rights (SWR)

\vspace{1em}
\textit{Additionally, the ``Roadmap to Reopening with Justice and Equity” \\is endorsed by the following organizations:}

Queer Student Resources (QSR)
\\Fossil Free Stanford
%\end{center} 

\vspace{2em}\hline

\justify\small{
Our coalition would like to thank the Black Law Students Association (BLSA) and the Stanford Black Postdoc Association (SBPA) for their advocacy on racial justice and police abolition issues this past spring and summer and for their involvement in the early stages of this project. Their campaigns have made much of our present work possible. We would also like to thank Hank Gerba, whose designs brought the Reverse Town Hall to life.
}

\justify\small{
% \centering\small{
Stanford University sits on the ancestral lands of the Muwekma Ohlone. We recognize Stanford’s continued presence on these lands---and, by extension, our own presence on these lands---as an ongoing occupation. As uninvited guests in this territory, we seek to move toward a way of being in and with this place that is light-footed, and that gives more than it takes. We honor the Muwekma Ohlone, past and present, and we commit to continuing to work toward ending and transforming the local patterns of exploitation, land theft, displacement, and settler colonialism in which Stanford is complicit.
}

\vspace{0.5em}
\justify {
% \centering {
Looking further to the south, we encourage all those who are able to support the \href{http://www.protectjuristac.org/}{Protect Juristac} campaign, aimed at preserving a landscape in Santa Clara County sacred to the Amah Mutsun Tribal Band.}
\vspace{-1em}
}

% \renewcommand*\contentsname{Summary}
\renewcommand*\contentsname{We call on Stanford to:}
% \renewcommand{\familydefault}{\sfdefault}
%\setlength{\parskip}{1em} 

\begin{document}
\thispagestyle{empty} 
\includepdf[pages=-,pagecommand={},width=1.31\textwidth]{cover_page.pdf}

\maketitle
% \vspace{0em}
% \begin{center}  
% \large
% The “Roadmap to Reopening with Justice and Equity” is a coalition-led effort \\spearheaded by the following organizations:
% \vspace{1em}
% Abolish Stanford
% \\Black Graduate Students Association (BGSA)
% \\Stanford Neighborhood Accountability Coalition (SNAC)
% \\Graduate Student Council (GSC)
% \\Sexual Violence Free Stanford (SVFS)
% \\Stanford Basic Needs Coalition (BNC)
% \\Stanford Solidarity Network (SSN)
% \\Students for Workers’ Rights (SWR)
% \vspace{1em}
% Additionally, the “Roadmap to Reopening with Justice and Equity” \\is endorsed by the following organizations:
% \end{center} 
% \newpage
\setlength{\epigraphwidth}{.93\textwidth}

\epigraph{\justify{
Historically, pandemics have forced humans to break with the past and imagine their world anew. This one is no different. It is a portal, a gateway between one world and the next. 
\vspace{2mm}
\\We can choose to walk through it, dragging the carcasses of our prejudice and hatred, our avarice, our data banks and dead ideas, our dead rivers and smoky skies behind us. Or we can walk through lightly, with little luggage, ready to imagine another world. And ready to fight for it.}
}{\vspace{0.25em} \textit{-Arundhati Roy}}

% \vspace{1.5em}
We are present together during a heart-wrenching and charged moment in history. The protests that have followed the shooting of Jacob Blake and the killings of Breonna Taylor, George Floyd, and countless other Black lives lost to racist and state-sanctioned violence, have foregrounded the white supremacist and genocidal foundations upon which this country and its law enforcement agencies were constructed. In recent weeks, we have watched as those who have taken to the streets to articulate a vision of a more just and equitable world have faced an increasingly fascist and militarized police force open about its collusion with vigilantes and have borne the full weight of institutions designed to uphold and reinforce white supremacy since their inception. Meanwhile, an ongoing global pandemic has reshaped our lives and our society over the course of mere months, disproportionately impacting Black, Native and Latinx communities and claiming the lives of close to 200,000 in the United States. And in just the last month, wildfires sparked and stoked by climate change and decades of mismanagement of our natural resources have ravaged the American West. Our air is poisonous, our cities and towns are in flames, and tens of thousands have been forced to flee their homes.

% \vspace{.5em}
Far from being insulated from these compounding developments, institutions of higher learning such as Stanford have an ethical duty to take on the most complex and challenging issues in the world today. Now is the time for fearless leadership and uncompromising dedication to justice, to accomplish what may have only ever seemed impossible in more ``normal” times, and to provide a model for other universities to follow suit.

% \vspace{.5em}
The COVID-19 pandemic has exposed and exacerbated structural inequities across the world, and we can see the effects in our own backyard. There are members of the Stanford community who are suffering, whether as a result of the pandemic or as a byproduct of the affordability crisis that plagues the student and campus community. Now is the time to center care in decision-making processes and to radically reimagine Stanford’s possibilities as an institution of higher education and as one of the economic engines of its community. 

% \vspace{.5em}
We come together as vital members of Stanford University. We come together in this moment to ask who we are as a community, who we are within this institution, and what we think this University should be and represent. We come together to ask if this University has the courage to create a culture of care and support for its students and workers, to re-envision public safety, to ensure that physical and mental healthcare is accessible for all, to be a good neighbor, and to transform complicity in white supremacy into institutional anti-racism. 

% \vspace{.5em}
This document assembles the work of over ten different student organizations, representing the voices of hundreds of undergraduate and graduate students and postdocs, whose collective efforts have sought input from thousands of members of the Stanford community. It is not a list of demands to be brushed aside as unrealistic, overly expensive or out of alignment with the University’s priorities; it is rather the vision of a University that centers care, equity and justice as its core principles. We, as members of this academic community, make our voices heard as key stakeholders in issues that must be addressed with gravity and urgency.

% \vspace{.5em}
What we demand, as the University enacts its plans to reopen its operations and restart research amid so much uncertainty, is for Stanford to be a just and equitable institution that reimagines community safety, prioritizes and centers Black Studies, repairs harm it has caused its neighbors, invests in the health and well-being of its grad students, advocates for undocumented and noncitizen members of its community, treats its workers with dignity, uses its resources to ensure that the basic needs of its most vulnerable students are met, and places the needs of survivors of sexual violence over its own reputation. We approach this process hoping that it will be a productive collaboration with University leaders, knowing that we as students and teachers have a critical role to play in helping Stanford live up to its legacy as a world-leading institution. 

% \vspace{.5em}
We also recognize the inherently interconnected nature of all of these pieces of our collective vision, and we know that our work will not---and cannot---be complete until the administration has agreed to implement each one of these pieces. Though the form of a roadmap suggests a temporal and spatial order, the arrangement of this collective vision statement does not reflect any pattern of prioritization or statement of comparative value on these issues. We do not lead single-issue lives, and all of these issues must be centered and acted upon immediately by the University. As a coalition, we stand behind  each and every one of them and speak in solidarity. In the words of Assata Shakur, we will \textit{love and support each other}, and future generations of Stanford students, until we have arrived at the last stop on this Roadmap.

% \vspace{.5em}
\begin{center}What follows is our outline of the steps that must be taken in order to actualize this vision.\end{center}

\newpage

\tableofcontents

\newpage
\section*{Defund and Disarm Campus Police.}
\addcontentsline{toc}{section}{Defund and Disarm Campus Police.}
\vspace{0.3em}
\hline\hline

We stand in solidarity with the George Floyd Rebellion. We amplify the measures articulated in the \href{https://docs.google.com/document/d/1clNfdcQWu4AHZ4HSB4x4dzddgpRXy_I3bJf9BTY3Adw/edit?usp=sharing}{Letter to Stanford President Regarding Policing on Campus} and the long-term aim of pursuing total abolition of police, prisons and ICE at Stanford and across the peninsula. We seek to build a coalition with all who oppose racialized state violence in its many forms, from the specific apparatuses of policing, surveillance, prisons, and detention centers to the systemic forces of colonialism, imperialism, and capitalism. Racialization and discrimination, institutionalized by the University’s exclusionary practices and reliance on state violence, have created an atmosphere of terror for Black students on campus. Stanford can, and must, do better. We invite the Stanford administration to join us in this struggle for liberation, to go beyond the one-year deferral of Stanford University Department of Public Safety (SUDPS) union contract negotiations and to seek a transformation of ``public safety” on our campus from one of policing, surveillance, and profit extraction to one of trust, care, and reparation.

Building a better world will take work from our whole community, but it can only be done if we free ourselves from the constraints outlined above. We need a world where we look after one another, where we take care of ourselves and others, where we all are equal. We need a world where justice is restorative and transformative, not punitive, and where our hopes and dreams are realized, because our entire University community helps us to realize them. We need a world where Stanford is a leader, not just in academics, but also in community building. To do this, we need to be in a world that is based not in violence, but rather in creation and care. Carceral measures have no place in this world; as such, it is a world that depends on the abolition of police and campus security.

The steps below help realize our vision for an abolitionist University that supports the safety, health, and humanity of every member of our community.

\subsection*{Reparate.}
\addcontentsline{toc}{subsection}{Reparate.}%
SUDPS was involved in the murder of 20-year-old Pedro Calderon on Stanford’s campus in 2002. Pedro’s family and community in East Palo Alto still have not seen any semblance of justice or reparations from Stanford or any of the governmental entities involved in the murder or its investigation. We call on Stanford to seek justice for Pedro Calderon, to provide reparations to his family and the community of East Palo Alto and to ensure no one is ever again murdered by police on our campus or in our broader community.

\subsection*{Dismiss and Disarm.}
\addcontentsline{toc}{subsection}{Dismiss and Disarm.}%
Stanford must open negotiations on, and ultimately end, the \href{http://sccgov.iqm2.com/Citizens/FileOpen.aspx?Type=4&ID=48992}{memorandum of understanding} between the University and the Santa Clara County Sheriff’s office that deputizes SUDPS officers to carry firearms and benefit from the same protections as county peace officers. The MOU effectively provides Stanford’s private police force the immunity of a public police force, but without public accountability, and thus sets a dangerous precedent for ``public-private” policing. Moreover, Stanford must terminate the collective bargaining agreement with the Stanford Deputy Sheriffs’ Association, which contains countless \href{https://docs.google.com/document/d/1v-WZPdZZnrDKqexZeMun8nixbeN0O6rEHawJTUglfBM/edit?usp=sharing}{terms and glaring omissions} that also shield Stanford deputies from any accountability. Until complete elimination of the MOU and collective bargaining agreement, negotiations for both must include student and subcontracted worker representatives with substantive veto power. These representatives must be those who are most negatively impacted by campus policing. 

\subsection*{Defund and Divert.}
\addcontentsline{toc}{subsection}{Defund and Divert.}%
Stanford should end its contract with SUDPS, and to thereafter continually decrease the police budget and allocate that money to avenues that nourish our community rather than punish it, including sexual violence prevention and support for survivors, funding explicitly for Black scholars and study, a fund to replace stolen property, financial aid and basic needs for all community members, and resources for the surrounding communities impacted by Stanford-driven gentrification. Funds can likewise be reallocated to hire Confidential Support Team (CST) counselors specifically to support Black and LGBTQ+ survivors, and to provide legal aid for survivors seeking justice through the Title IX office or auxiliary processes. Funding for sexual violence prevention work, as well as support for survivors through the CST and legal aid, will do more to keep our community safe than SUDPS has ever done. 

\subsection*{Desist and De-escalate.}
\addcontentsline{toc}{subsection}{Desist and De-escalate.}%
All community members should be trained in de-escalation and in alternatives to calling the police. Such training could be readily incorporated into alcohol, sexual harassment response, and safety trainings that already exist for incoming community members. We also advocate for a Stanford-specific switchboard for emergency calls that will route calls to an appropriate non-carceral crisis response.  

% \vspace{.5em}
\subsection*{Disclose.}
\addcontentsline{toc}{subsection}{Disclose.}%
Stanford and SUDPS should immediately make publicly available its policies on use of force, including deadly force, as well as ethical behavior, de-escalation protocols, misconduct investigation protocols, department budget, and agreements with the city of Palo Alto and Santa Clara county. Stanford should also release data on officer records, including misconduct; racial and gender-based profiling in stops, citations, and arrests; and outcomes of investigations into misconduct. Finally, students with veto power should be included in all decision-making bodies related to policing on campus.

% \vspace{.5em}
\subsection*{Discontinue.}
\addcontentsline{toc}{subsection}{Discontinue.}%
Despite student calls to defund campus police, Stanford has increased policing and surveillance on campus since June under the guise of legal compliance and public health. Policing is not a viable public health strategy. Stanford must end violent approaches to COVID-19 compliance response. The hiring of additional security for COVID-19 compliance increases the campus population and the risk of disease transmission. Increased policing also increases the likelihood of racial profiling. University representatives have deflected questions about SUDPS harassing students in the name of Campus Compact enforcement by stating that the police are obliged to respond when called, but the University has not advised the community against calling the police for minor Compact violations, or taken any measure to reduce police presence on campus. Thus, the University has tacitly condoned police harassment and relies on it to enforce the Compact through fear. Further, punitive compliance measures like termination of employment or eviction will fall most harshly on international students and those with the least generational wealth, the same victims already bearing the brunt of COVID-19 infection and fatality rates. 

Stanford must also cut ties with corporations that fund police specifically to terrorize East Palo Alto (\href{https://www.vice.com/en_us/article/d3akm7/how-facebook-bought-a-police-force}{Facebook}) and those that produce surveillance technology and sell it to police departments and federal government agencies (including but not limited to Amazon and Palantir). The University must disallow such corporations from recruiting on campus, and the University, its departments, and its research groups must cease to accept funding from and end partnerships with corporations that profit from the exploitation, surveillance, and murder of Black people, immigrants, and the working class as a whole. 

\subsection*{Trust.}
\addcontentsline{toc}{subsection}{Trust.}%
We propose an alternate model of peer education and bystander intervention that encourages support and engagement rather than surveillance and policing. The support networks that exist between people and communities at Stanford are precisely the means by which the responsibility to prevent the spread of COVID-19 is shared and upheld; top-down decrees and the threat of punitive measures do not serve the best interests of the community. A true public health approach to the pandemic must include freely available COVID-19 testing for \textit{everyone} who lives or works on campus (including and especially subcontracted workers who are often in highest-risk environments), as well as financial and medical support when any community member (including subcontracted workers) needs to recover, support loved ones, or quarantine. With sufficient testing and sufficient medical and financial support, our community can keep itself healthy without police surveillance or threats to employment and housing.

\newpage
\section*{Prioritize Black Studies, Departmentalize AAAS.}
\addcontentsline{toc}{section}{Prioritize Black Studies, Departmentalize AAAS.}%
\vspace{0.3em}\hline\hline

Despite the rich history of the Stanford Program in African and African American Studies (AAAS), which celebrated its 50$^\mathrm{th}$ anniversary in 2019, the program remains underfunded and understaffed. AAAS struggles to cobble together a coherent educational experience for its 47 majors and minors. For AAAS, high enrollments have not translated into higher visibility or higher prioritization within a University that has nevertheless expressed a commitment to fighting against anti-Blackness both on- and off-campus. Stanford’s inaction indicates a devaluation of the theoretical and empirical contributions of Black Studies scholars globally. It is not enough to say that there is much work to be done. Our current moment demands that we all do the work, and do it now. Our vision for AAAS is to be a nationally-recognized leader in teaching, researching, and producing scholarship in the field of Black Studies. In order to achieve these goals, we submit the following:


\subsection*{Departmentalization.}
\addcontentsline{toc}{subsection}{Departmentalization.}%

% \begin{itemize}
% \renewcommand{\labelitemi}{--}
% \item
\textit{The University must outline a clear path to departmentalization for the African and African American Studies program in the next five years.} Departmentalization will guarantee the sustained ability to support undergraduate and graduate students and grant academic autonomy in matters of hiring and curriculum development, which have direct bearing on future success. 
% \item 

\textit{This path to departmentalization should include a cluster hire of faculty trained in Black Studies Epistemology, as well as sustained funding for the department} so that intra-departmental resources no longer impede African and African American Studies from fully supporting its students.

% \end{itemize}

\subsection*{Urgent Support for AAAS Undergraduate and Graduate Students.}
\addcontentsline{toc}{subsection}{Urgent Support for AAAS Undergraduate and Graduate Students.}%



Recognizing that the immediate needs of students and staff must be addressed before the goal of departmentalization is actualized, \textit{financial resources should be made available to support the following urgent needs while developing a plan for departmentalization within the 2020-2021 Academic School Year}:

\begin{itemize}[itemsep=1em]
% \renewcommand{\labelitemi}{--}
\item{\textbf{Lines of funding for two Postdoctoral Fellowships in AAAS}. Postdoctoral fellows do the important work of meeting student demand, teaching key courses, and creating an academic program befitting of Stanford’s stature. Postdoctoral fellows in AAAS demonstrate Stanford’s commitment to producing groundbreaking scholarship, diversifying the faculty, bringing rising star scholars to campus, and addressing the “pipeline problem.”}

\item{\textbf{Convert the current Research Fellow appointment (One-year fixed-term position) into a permanent appointment (three-year position with the possibility of renewal)}. The AAAS Research Fellow has played a critical role in the development of programming for graduate students including the Emerging Scholars Conference. This appointment is also critical to increasing research opportunities and mentoring/advisement support for undergraduates.}

\item\textbf{The appointment of a designated Development Officer} to secure resources from donors and other institutions interested in supporting AAAS initiatives.

\end{itemize}% \end{itemize}

\newpage
\section*{Repair Harm Done to Local Communities.}
\addcontentsline{toc}{section}{Repair Harm Done to Local Communities.}%
\vspace{0.3em}
\hline\hline

Stanford bears exceptional responsibility to acknowledge and repair the harm that it has caused to surrounding communities since its founding. Stanford was founded by a white supremacist on land that was violently stolen from the Muwekma Ohlone people, using wealth and power amassed through the genocide of Native peoples and exploitation of indentured Chinese laborers. Since then, Stanford has been a reckless and immoral neighbor, but most people do not realize the extent of the harm the University has inflicted and continues to inflict on surrounding communities. Today, Stanford is \href{https://extras.mercurynews.com/whoowns/stanford.html}{Silicon Valley’s largest property owner, with total land value greater than that of Google, Apple, and Intel combined}. Despite being one of the primary drivers of gentrification and displacement in the Bay Area, Stanford refuses to adequately mitigate the impacts of its continued expansion. In 2019, Stanford withdrew its General Use Permit application, evading its responsibility to provide affordable housing for workers and to consider the needs of surrounding communities, especially East Palo Alto, East Menlo Park/Belle Haven, and Redwood City/North Fair Oaks. Moreover, Stanford’s attempts to carry out local outreach have been under-resourced at best, and exploitative and self-serving at worst. Stanford’s lack of investment in surrounding communities is unacceptable. In taking earnest first steps to become a better neighbor, Stanford must:  

\begin{itemize}[itemsep=1em]
    \item \textbf{Commit to explicit acknowledgement of harm, reparations for, and accountability to the Muwekma Ohlone people, the rightful owners of the land Stanford occupies.} The renaming of Serra Mall and Serra House is not nearly enough. The following additional measures are warranted:
    \begin{itemize}
        \item Work with the Muwekma Ohlone to center initiatives they value, such as previous efforts to \href{https://news.stanford.edu/2019/11/06/stanford-prepares-rename-jane-stanford-way-honors-relationship-muwekma-ohlone/}{repatriate remains} and the \href{http://www.muwekma.org/tribalhistory/recognitionprocess.html}{ongoing struggle for recognition}, on their terms and to the extent they deem appropriate and useful.
        \item Begin each Stanford campus tour by describing the ruthless history of how Leland Stanford accumulated his wealth, and by acknowledging the Muwekma Ohlone as the rightful stewards of this land and the enslavement and genocide that they faced at the hands of colonizers.
        \item Offer to pay a \href{https://sogoreate-landtrust.com/shuumi-land-tax/}{land tax} to local Ohlone peoples in deference to, and support of, their ongoing cultural heritage here.
    \end{itemize}

    \item \textbf{Offer reparations for harm inflicted on surrounding communities and investment in their efforts towards affordable housing, equitable education, and quality healthcare.} \href{https://bcsc.stanford.edu/sites/g/files/sbiybj10366/f/justice_and_hope-_scanned.pdf}{For 50 years}, students have advocated for reparations to low-income communities in East Palo Alto, East Bayshore, and East Menlo Park (Belle Haven) to compensate for displacement and discrimination by the University. The same communities have been excluded from negotiations with Santa Clara County on future expansion. Stanford must meet its obligation to repair the harm it has done to our neighbors throughout its history through ongoing investment in housing, education, and healthcare for our neighbors in Santa Clara County \textit{and} San Mateo County, particularly in the communities of East Palo Alto, Belle Haven, and North Fair Oaks.
    
    \item \textbf{Provide affordable housing and transportation benefits for ALL members of the Stanford community, including workers, commensurate with the University’s projected growth.} Stanford workers are essential to our community, but most---especially subcontracted workers---are denied a living wage, affordable housing and transportation, and other essential benefits. We follow the leadership of SCoPE 2035 and Students for Workers’ Rights in advocating for the dignity and just treatment of \textit{all} workers on campus. Specifically, Stanford must:
        \begin{itemize}
            \item Provide free parking on campus for subcontracted workers who have the lowest wages of all University affiliates and who often have to commute hours each way due to the extreme unaffordability of the region.
            \item Commit to \textit{serious} expansion of on-campus housing available to workers: at minimum, as many units as the number of people who have requested to be on the waiting list for Stanford West apartments. 
            \item Add a free Marguerite line through East Palo Alto, with the same stops that were used in the pilot shuttle cancelled by Stanford at the last minute.
        \end{itemize}
    
    \item \textbf{Support local communities during emergencies.} As wildfires rage throughout the Bay Area amidst an ongoing pandemic, low-income communities and communities of color continue to be at greatest risk of adverse \href{https://www.phi.org/press/wildfire-smoke-poses-greatest-risk-to-low-income-residents-people-of-color-experts-say/}{health outcomes} and \href{http://latogether.org/2020/08/26/calfires/}{economic burdens}. These same communities have borne the brunt of Stanford’s selfish and exploitative practices for decades. We call on Stanford to fulfill its mission to deploy its housing, medical, legal, and financial resources to benefit the region, including and especially during social and environmental emergencies. A few concrete ways in which Stanford can support local communities include: 
    \begin{itemize}
        \item At minimum, matching \href{https://charity.gofundme.com/o/en/campaign/covid-19-relief-for-muwekma-ohlone-families?fbclid=IwAR17ULCWJcv0rqeQQOFQ71UY3zhUmLTiC1Bivzro2KvVh4e2mJ-fq_E8y2E}{this fundraiser} for direct COVID-19 relief for Muwekma Ohlone families.
        \item Providing hazard pay to essential workers at Stanford, including subcontracted workers.
        \item Following the lead of San Jose State in providing accommodation and support for wildfire evacuees.
    \end{itemize}

\end{itemize}

\newpage
\section*{Support Grad Students in the Age of COVID-19.}
\addcontentsline{toc}{section}{Support Grad Students in the Age of COVID-19.}%
\vspace{0.3em}\hline\hline

\subsection*{Housing.}
\addcontentsline{toc}{subsection}{Housing.}%

Housing is a major source of anxiety for graduate students. With the opening of the new Escondido Village Graduate Residences this fall, rent increased 16\% across all graduate housing options for the 2020-21 school year, while the availability of “affordable housing” (that is, housing that costs less than 30\% of income) decreased by 80\%. Despite Stanford’s purported concerns about affordability and the formation of the \href{https://www.stanforddaily.com/2020/02/19/after-1-5-years-all-stanfords-affordability-task-force-has-to-offer-are-false-solutions/}{Affordability Task Force}, the amount of graduate housing that costs less than 40\% of income will also decrease by 3\% from the previous year, while the availability of housing costing 40-60\% of income is set to increase by 119\%. As a point of reference, the U.S. Department of Housing and Urban Development has found that \href{https://www.huduser.gov/portal/pdredge/pdr_edge_featd_article_092214.html}{those who pay more than 30\% of their income on housing are considered cost-burdened} and may have difficulty affording necessities such as food, clothing, transportation, and medical care. That is the reality for almost every graduate student living in University housing on a graduate stipend while teaching classes, conducting research, and trying to make ends meet in the Bay Area. In order to mitigate the real financial burdens associated with living in campus housing, we advocate for the amendment of graduate student housing policies and rent in the following ways:

\begin{itemize}[noitemsep]
    \item Extend graduate student housing priority by one year;
    \item Taking into account the fall in local housing prices due to COVID-19, decrease rent to rates commensurate with 30\% of graduate student stipends;
    \item Revert the content of the 2020-21 residence agreement to that of the 2019-20 agreement, removing added stipulations on Class Action Waiver, Force Majeure, and Reporting Communicable Diseases;
\end{itemize}

In addition, the Campus Compact should be amended to:
\begin{itemize}[noitemsep]
    \item Halt all evictions of students from campus housing in accordance with County and State guidelines; 
    \item Allow students, especially those living alone, to form “pods” in order to avoid the mental health strains of isolation, continuing to work and advocate with the County government to ensure that this becomes possible;
    \item Subsidize costs of single-living spaces for students at high risk of complications from COVID-19;
    \item Commit to daily reporting of available isolation space for infected individuals;
    \item Meet the essential needs of students in isolation by providing food, medications, essentials to students in isolation spaces via a specifically designated fund of the University’s emergency management funds.
\end{itemize}

\subsection*{Healthcare.}
\addcontentsline{toc}{subsection}{Healthcare.}%


The past few months have only served to reinforce the importance of access to healthcare not just for individuals, but for the health and success of the whole community. At Stanford, access to that healthcare places a significant financial burden on graduate students, \href{https://static1.squarespace.com/static/5a371f50bff200aa91b5113a/t/5ac1dcbb8a922d9f466b040d/1522654403498/GSC-DAC+2017-18+Survey+Report.pdf}{with thousands of students paying up to 17\% of their total annual stipend} on health-related expenses. In the midst of a global pandemic that is endangering the lives of everyone in the Stanford community, this cost burden for the basic necessity of healthcare is unacceptable. As it now stands, members of the Stanford community do not have the same access to health care depending on their home department. This discrepancy is particularly egregious in light of the 12-month funding guarantee across all departments. If Stanford accepts that all graduate students deserve equal compensation for their work, then we assert that Stanford must likewise provide complete and equal healthcare coverage for all of its graduate students. A standardized policy across all Schools and departments that fully covers Cardinal Care for all graduate students would simplify insurance negotiations on the administrative end and decrease harmful inequities between students in different departments and programs.

\subsection*{Mental Health.}
\addcontentsline{toc}{subsection}{Mental Health.}%

The current mental health crisis predates 2020, which saw dramatic spikes in reports of distress, isolation, anxiety, depression, and suicidal ideation among college students across America and traditional models falling short leaving student needs unmet (Sarah Brown, \textit{Overwhelmed: The Real campus mental-health crisis and new models for well-being}, Chronicle of Higher Education, 2020). Stanford is no exception: we all remember the alerts in the AY 2018-2019 announcing that yet another community member had taken their own life, and most Stanford students know from experience that CAPS has unreasonably long waitlists for therapy, that we cannot count on follow-ups for off-campus referrals, that the cost of co-pays for in-network care are prohibitive and the cost of out-of-network care even more so, and that non-clinical support resources remain limited. It should come as no surprise that students have struggled to access quality mental health care since March. The present crisis warrants both an \textbf{increase in funding for mental healthcare resources} for Stanford students, and \textbf{further advocacy on behalf of Stanford students} where barriers to access remain. 

Increased funding will allow for: 
\begin{itemize}[noitemsep]
    \item The creation of a Mental Health Care Fund, separate from Emergency Grant-in-Aid, which explicitly provides reimbursement for both emergency and continued care for services including, but not limited to: therapy, prescription medication, and psychiatric care provided in-network, and, in special circumstances, out-of-network (those circumstances may include: lack of availability of CAPS clinicians or in-network providers; students living out of California due to travel and/or immigration restrictions).
    \item The hire of additional CAPS clinicians in order to increase the availability of mental healthcare services and improve continuity of care where students are allowed an unlimited number of appointments with CAPS clinicians, with an explicit emphasis on increasing the diversity of available providers;
    \item Programming accessible, low-barrier, and virtual CAPS services (modeled on CAPS Connect).
\end{itemize}

\newpage
Advocacy on behalf of students should come in the form of:
\begin{itemize}
    \item Support for the ASSU campaign for granting telehealth licensing across state lines; 
    \item A campaign to eliminate stigma around mental healthcare for graduate students; 
    \item Greater transparency around the resources available to all students, particularly out-of-state and students living abroad, as well as wait times for phone screenings, intake appointments, and regular appointments, including for providers holding different identities or specialties;
    \item Collaboration with CAPS for the creation of a cancellation waitlist system that allows students to be seen at the last minute if someone else cancels their appointment. 
\end{itemize}

\subsection*{Support for Students with Dependents and Families.}
\addcontentsline{toc}{subsection}{Support for Students with Dependents and Families.}%

Over the past six years, Stanford has raised dependent health insurance rates by 80\%, far exceeding any stipend or pay rate increases. Some student workers with children pay over a third of their income to keep their children insured. The exorbitant costs of Stanford’s plans have forced many students with dependent children onto public assistance. Stanford’s reliance on Covered California and MediCal to meet the health care needs of its student workers strains state and federal resources. Stanford is one of the most resource-rich institutions in the world. Forcing student workers onto public assistance because the rates of dependent health insurance are unaffordable places an unfair burden on public resources that could be used to support other community members in need. Stanford should provide more affordable healthcare to all dependent children and spouses in the Stanford student community and to support graduate student families through:
\begin{itemize}
    \item An emergency enrollment period for the dependent health care plan, with the aim of ensuring that all Stanford families have access to quality health care at this time;
    \item Further subsidies of dependent health care costs for spouses, and 100\% subsidized coverage for children;
    \item Disbursing Graduate Family Grant funds to compensate parents for the additional childcare labor they must now perform due to COVID-19.
\end{itemize}

\subsection*{Power Abuse.}
\addcontentsline{toc}{subsection}{Power Abuse.}%

During the shelter-in-place and throughout reopening, students working in research labs across campus have been \href{https://www.stanforddaily.com/2020/03/25/publish-and-perish-despite-shelter-in-place-grad-students-cite-pressure-to-continue-lab-work/}{forced to return} to work by their advisors. The acute risks of the pandemic have shone a spotlight on chronic abuse stemming from power imbalances between faculty and the researchers they employ. This dynamic allows for exploitation through threats, spoken or unspoken, to financial, career, and physical safety. Trainee researchers are highly dependent on their advisors for funding (and by extension, access to healthcare), visa status, and their onward career success. Fear of retaliation has directly led to numerous unreported instances of sexual harassment, safety violations, and physical threats, perpetuating the cycle of abuse. Currently, there are no effective ways to address these abuses of power, and the offices and processes that are available to trainees are not well publicized and difficult to understand. 

The situation requires a new system of handling faculty harassment and abuse in a way that provides for the protection of trainees while holding abusive faculty members accountable. Such a system will pave the way for a safe place for trainees to flourish, especially benefiting those with marginalized identities, empowering us to better learn and contribute to our institution and world.


\textit{Preventative Measures
}

\begin{itemize}
    \item For accountability and transparency, the University must lay out clear, accessible, and consistent policies on harassment and standards of behavior (including microaggressions and power-based harassment), consequences for violations and retaliation, as well as protections for complainants. These policies --- a “Trainee Bill of Rights” --- must clearly delineate the scope of what can be asked of graduate student workers and what constitutes abuse. 
    \item The voices of student and staff workers should be included in major department decisions, including but not limited to admissions, faculty hiring, department chair selection, and development of departmental policies for students and faculty. 
    \item Each School in the University should administer a centralized, anonymous climate survey for their respective departments to measure the effective department culture among its various constituents. The School must then be responsible for ensuring departments rectify harmful behavior. 
    \item The University should provide regular, detailed reporting of statistics and outcomes of the reporting process and the climate and mentorship surveys. These data should be aggregated and anonymized to protect the complainant. 

\end{itemize}

\textit{Reactive Measures and Grievance Processing}

\begin{itemize}
    \item The University must create avenues for anonymous reporting of abusive actions by members of the Stanford community, including faculty, administrators, and public safety officials. This must be a centralized reporting system (such as \href{https://secure.ethicspoint.com/domain/media/en/gui/7325/index.html}{EthicsPoint}, which is employed by peer institutions) that allows for the filing of both formal and informal grievances. There should be no time limit following the incident to file a grievance.
    \item The University must provide a clear and consistent outline with detailed steps about these avenues (as well as Title IX procedures) so that all student and staff workers are aware of how each step of the process will work and what will be required of them before they begin the grievance process.
    \item There must be clearly designated procedures in place to protect students from retaliation from the accused party. These should be specified to include (but are not limited to) threats to personal health and stability, threats to academic career prospects, coercion to prevent the filing of grievances, or dismissal. Retaliatory incidents should be treated with the same seriousness as the original reported incident and continually monitored throughout the affected party’s career at Stanford. 
    \item The University must provide support resources, such as restorative mediation, a secondary mentor, transitional funding, and academic accommodations to facilitate the ability for the harassed party to leave a dangerous and/or dysfunctional situation. This must be provided officially and unilaterally across all schools and departments. 
    \item Grievances received through the centralized reporting system described above must be analyzed by an advisory panel that will investigate the accusation and determine possible recourse, including punishment if necessary. This advisory panel should include students, postdocs, and staff members in addition to faculty members.
\end{itemize}

\textit{Faculty Accountability Measures}
\begin{itemize}
    \item All faculty will go through mandatory mentorship training (e.g. UC Berkeley’s “Advancing Practice”), which should include conflict resolution training, anti-racist and anti-bias training, and effective mentoring across identities.
    \item Each faculty member must go through a mandatory yearly supervisor assessment with each of their mentees before they can take on any new students. These assessments should be evaluated both by fellow faculty members and student and staff workers. Any faculty member that fails to be found a suitable mentor will be required to have additional training and quarterly assessments, after which continued failure will result in revocation of mentorship privileges. 
    \item If climate surveys or reporting avenues show repeated instances of harassment and abuse within a single department, the department should be held accountable and have its culture examined by an external board. This review should be conducted by neutral, qualified appointees from the School to oversee the appropriate recourse.
\end{itemize}

\subsection*{Extension of Funding.}
\addcontentsline{toc}{subsection}{Extension of Funding.}%


The pandemic has affected all areas of graduate research for students in every PhD program. For the past six months, COVID-19 has put research projects on hold, increasing graduate students’ concerns related to publication, promotion, and progression in their academic careers. Faculty, many of whom are taking on additional childcare tasks, are less able to offer advising. Additionally, the cancellation of conferences and colloquia has curtailed opportunities for PhD students to present research, receive feedback, and establish themselves as scholars in academic fields. 

Looking forward, many universities and other organizations have publicly announced “hiring freezes,” casting uncertainty upon large portions of the academic job market as the global economy heads towards recession. Stanford has already recognized the impact of the pandemic on career progression in granting a \href{https://facultyhandbook.stanford.edu/covid-19-health-emergency-tenure-clock-extension}{one-year tenure clock extension} to faculty, but the disruption that the COVID-19 pandemic has impacted career progression at all levels. In addition to graduate students, postdoctoral scholars, integral members of our research and teaching community, face these same conditions. \href{https://www.pnas.org/content/116/42/20910}{Peer-reviewed research} has demonstrated that postdocs have more of an impact on graduate student skill development than PIs. We envision a University that works to mitigate the negative career impacts of the pandemic for all members of its community, from graduate students to postdocs to faculty. 

Peer institutions have already taken steps to support graduate students with funding extensions. Harvard’s Graduate School of Arts and Sciences, for example, has allowed students to apply for an extra year of “\href{https://gsas.harvard.edu/news/stories/gsas-emergency-support-initiative}{lost-time funding}” if they can demonstrate that their research was impacted by the pandemic. Yale’s Graduate School of Arts and Sciences is offering \href{https://gsas.yale.edu/news/deans-message-about-funding-extensions-students}{funding extensions} of an academic term or a year on top of its existing programs to guarantee sixth-year assistantship funding to humanities and social science students. To keep the Stanford name competitive and maintain our placement records in academia, we must at least match the safeguards of our peers. The School of Humanities and Sciences recently announced that they will allow departments to fund students in their $6^{\mathrm{th}}$ and $7^{\mathrm{th}}$ year by reducing future cohort sizes; it is unclear at this juncture whether departments are willing to exercise this mechanism to protect their current students. 

In the midst of a global pandemic, funding stability is an urgent need for graduate students and postdoctoral scholars across the University. It is therefore essential that the University:  

\begin{itemize}
    \item Ensure that \textit{all salaried/stipend-earning graduate students} have access to a fully funded one-year program extension, granted upon request over the next three years.
    \item Match external funding sources that are canceled due to pandemic-related disruptions.
    \item Match any increases in rent, cardinal care insurance premiums, and inflation with an equivalent increase in pay rates. Pay rates should be adjusted such that rent remains at most 30\% of stipend levels.
    \item Offer an optional one-year funding extension to all postdocs from University funds (i.e. not PI grants).
\end{itemize}

\subsection*{Emergency Funding.}
\addcontentsline{toc}{subsection}{Emergency Funding.}%

Since the graduate student \href{https://medium.com/@stanfordsolidaritynetwork/5-demands-from-stanford-graduate-students-in-response-to-the-covid-19-crisis-7b03f6c50485}{letter to the deans} in May of this year, the economic crisis provoked by the COVID-19 pandemic has only worsened and the financial circumstances of many Stanford graduate students have become even more precarious. Moreover, the vast majority of us have been forced to relocate (in some cases, multiple times) since the start of the pandemic, meaning that many of us are facing financial precarity as the summer draws to a close. Recently the \href{https://medium.com/@stanfordsolidaritynetwork/stanford-promises-all-phd-students-five-years-of-guaranteed-12-month-funding-4c1cb5b2f412}{Stanford administration agreed to ensure summer funding for PhD students across all departments, in response to years of graduate student advocacy}; this decision, however, takes effect in AY 2020-21, meaning that those who were impacted by the cancellation of summer research or teaching opportunities and who were housing or food insecure in May will receive no relief from the financial strains of the pandemic. It is unconscionable to frame this necessary 5-year funding guarantee as a response to \href{https://news.stanford.edu/2020/07/23/stanford-commits-12-month-funding-phd-students/}{the increased vulnerability of graduate students during the COVID-19 pandemic}, when the relief provided by this guarantee will not benefit graduate students for a full year.

For these reasons, we believe that a just and equitable reopening is contingent upon:

\begin{itemize}
    \item \textbf{The immediate disbursement of a one-time grant in the amount of \$1,500} for all salaried/stipend-earning graduate students.
    \item \textbf{The expansion of Stanford’s Emergency Grant-in-Aid Fund to cover living expenses} for students in emergency situations.
\end{itemize}

\newpage
\section*{Advocate For International Students.}
\addcontentsline{toc}{section}{Advocate For International Students.}%

\vspace{.5em} % here bc otherwise there's a negative space? 
\hline\hline

Stanford has failed to adequately provide much-needed institutional support to its international students. In July, sudden and severe changes in Immigration and Customs Enforcement (ICE) policy plunged international students into panic, as they feared they would be deported or lose their stipends. Even as the University lobbied against these policies, the Bechtel International Center left international students in the dark, failing to communicate with them until the policy was reversed. Bechtel’s shortfalls extend beyond this singular emergency. International students do not feel that Bechtel consistently fights for their safety, security, or wellbeing, according to a recent \href{https://docs.google.com/forms/d/e/1FAIpQLSeRj57M0mXMKo2pvqEIhHN7VghqA_E0PUXkyu7Jvpn7wyOLuA/viewform?pli=1}{ASSU survey}. Instead, advocacy for international students has often been left to unofficial channels, such as the Immigrants’ Rights Clinic at the Law School, despite the fact that this is not the role of the Clinic, and that Law School faculty at the Clinic remain uncompensated for this additional labor.   

It is not enough to simply ask that Bechtel do more. Indeed, given its role in ensuring compliance with visa policy and directly providing student information to ICE, Bechtel cannot be perceived as a trustworthy, reliable ally to international students. International students deserve campus institutions, properly supported by Stanford, that understand their needs and advocate on their behalf, and that can provide them with timely, proactive communication and access to legal advice. To properly support international students, the University must:

\begin{itemize}
    \item Create an International Student Community Center, independent of Bechtel, whose sole mandate is to advocate for international and undocumented students on both immigration challenges and University-related issues. 
    \item Hire a full-time immigration attorney in the Center who can advise international students and provide direction for University policies to support students in anticipation of changes in immigration law.
    \item Hire staff at the Center and/or delegate staff around the University to advocate for international students in the creation of policy around academic affairs, student life, housing, and working conditions. While these staff may ultimately be employed in different offices across the University, it is key that they are able to report back to the international student body in a centralized fashion through the Center. 
    \item Advocate proactively, through the University’s lobbying team, at all relevant governmental levels on behalf of international students. This includes, but is not limited to, advocacy for: 
    
    \begin{itemize}
        \item the prompt reopening of visa services, 
        \item the long-term preservation of post-completion Optional Practical Training (OPT), which gives international students the ability to work in the US after graduation, 
        \item nondiscrimination against Chinese students and other targeted groups, and
        \item generous policies around preserving “student status” in light of ongoing online education and travel restrictions. 
    \end{itemize}
    
    Furthermore, this lobbying must be clearly publicized to international students so that they know that the University is working on their behalf.

\end{itemize}

\newpage
Secondly, it is imperative that Bechtel:
\begin{itemize}
    \item Communicate, in consultation with this new Center attorney, clearly and proactively in a timely manner to international students. This should be in a digestible form, such as an FAQ that describes changes to immigration policies, Stanford’s response, and relevant legal developments. Stanford must ensure that this information is distributed widely and consistently across all relevant webpages, social media channels and listservs. 
\end{itemize}

Thirdly, both Bechtel and the University should:
\begin{itemize}
    \item Consult the ASSU’s directors of International Student Advocacy and/or include student voices on committees in decision-making that will affect international students. 
    \item Pursue partnerships with international institutions through the Bing Overseas Studies Program to enable international students to continue gaining Stanford credit if unable or unwilling to return to the U.S. These programs can build infrastructure to assist students who might have difficulty advocating with their department to take advantage of the loosened transfer credit policy.

\end{itemize}

\newpage
% \section*{Advocate For Undocumented and Migrant Students.}
% \addcontentsline{toc}{section}{Advocate For Undocumented and Migrant Students.}%
\section*{Advocate For Undocumented \& DACAmented Students.}
\addcontentsline{toc}{section}{Advocate For Undocumented \& DACAmented Students.}%

\vspace{0.5em}\hline\hline

Stanford's undocumented, DACAmented, and mixed-status affiliates and their families have faced numerous hardships since the Trump administration took power in 2017. Their precarity was heightened by a new policy \href{https://www.dhs.gov/news/2020/07/28/department-homeland-security-will-reject-initial-requests-daca-it-weighs-future}{statement} put forth by the administration on July $28^{\mathrm{th}}$, 2020, which outlined new restrictions on the Deferred Action for Childhood Arrivals (DACA) program. As demonstrated by the success of Stanford’s support for the lawsuit filed by MIT and Harvard against the government’s visa policy change for international students, it is clear that the University’s actions hold great weight in fighting harmful policies and sending a supportive message to its student body. To date, Stanford has remained silent regarding the government’s policy change. Stanford must stand up for its DACA/undocumented community and their families and support them in this difficult time. This support should come in the form of:
\begin{itemize}
    \item An unambiguous statement to our Stanford undocumented and DACAmented community, stating that \textbf{legal action} will be taken on behalf of DACAmented and undocumented students against these new restrictions as the first step to abolishing ICE. In this statement, Stanford must also commit to supporting the American Dream and Promise Act (H.R. 6) and further legislation for all 11 million undocumented migrants. Further, Stanford must also acknowledge the historical atrocities of ICE and DHS, recognize the systems of oppression that allow for such governmental agencies to exist, and commit to abolishing these institutions. 
    \item A task force designed to implement and enforce training for all faculty, staff, and administration in order to better support undocumented students. This task force must be vetted, maintained, and overseen by a board of students, staff, and faculty.
    \item Hiring of full-time professionals of color who are better trained on working for DACAmented and undocumented students, as well as more resources to increase awareness around issues related to supporting undocumented students.
    \item Adequate information for undocumented students. Stanford must regularly update \href{https://undocumented.stanford.edu/}{undocumented.stanford.edu} and include additional relevant resources and information in New Student Orientation week booklets and packets for new and incoming students. These updates and resources must be in consultation with undocumented students and activists. 
    \item Substantial financial support for undocumented students. Stanford must prioritize undocumented students in matters of financial aid packages, including full financial aid scholarships, fee waivers, yearly stipends, and job-based stipend opportunities. The University must create a form on \href{https://financialaid.stanford.edu/}{financialaid.stanford.edu} dedicated to any additional needs for undocumented students to be reviewed by the FAO.
    \item Active commitment to ensuring the safety and wellbeing of our undocumented community in the form of professional staffing. The University must hire staff members, such as CAPS counselors, specifically trained in working with migrants and migrant trauma, and should assign dedicated administrative staff to work with the undocumented student population. 
    % \item Resources for workers, building on the recent work of Students for Workers’ Rights (SWR), including: 
    % \begin{itemize}
    %     \item increased funding for student organizations such as Habla who work directly with migrant workers and migrant communities;
    %     \item collaboration with local organizations, such as Upward Scholars, to help workers attain undergraduate degrees; and
    %     \item either access to outside legal counsel, or an increase in institutional support for the Immigrant Rights Clinic at Stanford Law School through the hiring of full time attorneys.
    % \end{itemize}
    \item Emergency grants for international and undocumented students to cover retention of private counsel. While always costly, the overlapping public health, economic, ecological and political crises have compounded the potential financial strain on these communities. Stanford must offer this support to students unable to afford necessary legal aid.

\end{itemize}

\newpage
\section*{Ensure the Safety of Subcontracted Workers.}
\addcontentsline{toc}{section}{Ensure the Safety of Subcontracted Workers.}%
\vspace{0.3em}\hline\hline


% \vspace{2mm}
The University’s ongoing failure to support subcontracted workers could have disastrous public health consequences. Stanford discourages subcontracted workers from reporting their symptoms through punitive COVID-related testing protocols, denies them rapid testing, which is accessible to other members of the Stanford community, and refuses to support subcontracted workers that fall sick. This comes alongside the University’s negligence in providing subcontracted workers with sufficient personal protective equipment (PPE), much less adequate knowledge about potential work-related SARS-CoV-2 exposure. Knowingly or unknowingly, Stanford is forcing subcontracted workers to choose between financial security and mitigating the spread of the novel coronavirus.

% \vspace{2mm}
% \noindent
Subcontracted workers are essential workers. Without their labor, Stanford would not be able to resume any in-person teaching or research activities. These workers have played a vital role in safeguarding the health of those still living and working on campus at great peril to themselves, their families, and their communities. Beyond their labor, essential workers are also integral members of our campus community: they are our friends, allies, and confidantes. Far from recognizing this, Stanford has laid off a significant portion of this workforce. Though the University pledged pay continuation for laid-off workers, it has broken its promise, failing to support the majority of subcontracted workers that have lost their jobs. What is more, those workers that remain have been burdened with increasingly demanding workloads, receiving neither an increase in regular wages nor hazard pay despite the high levels of risk that they face. 

% \vspace{2mm}
% \noindent
If Stanford wishes to keep essential workers safe, the University must implement the following measures:

% \vspace{-0.5em}
%\subsection{Covid-related healthcare guarantees}
\subsection*{COVID-Related Healthcare Guarantees.}
\addcontentsline{toc}{subsection}{COVID-Related Healthcare Guarantees.}%
%\begin{enumerate}
%\item Non-punitive COVID-related protocols:
\begin{itemize}
    %\renewcommand{\labelitemi}{--}
    \item
    \textbf{Removal of barriers to testing:} Rapid on-demand COVID testing must be accessible for all subcontracted workers during (paid) working hours. Currently, subcontracted workers do not have access to the rapid testing that is available to other members of the Stanford community. They are forced to travel long distances to testing sites across the Bay Area in their already limited days off work, sometimes having to wait weeks for test results. Removing these testing barriers will safeguard both subcontracted workers and the other members of the campus community with whom they come into contact.
    \item 
    \textbf{Quarantine pay:} If a worker tests positive for COVID-19 or is awaiting test results, they must receive full compensation during any mandated period of self-isolation. Subcontracted workers have been made to use personal sick and vacation days if they are required to self-quarantine. By forcing many workers to effectively forego pay, this policy disincentivized workers from disclosing symptoms and increases the likelihood that mild COVID-19 cases will go unreported. 
    \item 
    \textbf{Personal protective equipment:} All workers must have access to sufficient PPE, including medical-grade face masks. Though some PPE is currently available, the supply is often insufficient to provide for all workers. Subcontracted workers must also provide their own face masks. In not standardizing PPE provision across the workforce, this policy requires workers to either take on additional financial burdens to fulfill their workplace obligations, or risk personal health and the health of people around them. 
    \end{itemize}
%\item Just compensation for essential work:

\vspace{-0.5em}
\subsection*{ Just Compensation for Essential Work.}
\addcontentsline{toc}{subsection}{Just Compensation for Essential Work.}%
\begin{itemize}
    %\renewcommand{%\labelitemi}{--}
    \item
    \textbf{Back pay for laid-off workers:} President Marc Tessier-Lavigne \href{https://healthalerts.stanford.edu/covid-19/2020/05/27/a-message-from-president-marc-tessier-lavigne-our-financial-future/}{guaranteed pay} continuation for laid-off workers through August $31^{\mathrm{st}}$. \href{https://docs.google.com/document/d/1DHMi1mgQHcP8OVrczQE3t-BOZd-qp1Elte3Y1cTCtv0/edit}{75\% of these workers have not received any of these funds} since June $15^{\mathrm{th}$, nor have they had access to the health care that they were promised for this period of time. Stanford must fulfil this promise by paying these workers the back pay they are owed and by providing them with healthcare for a duration equivalent to the period during which they did not receive it. 
    
    \item
    \textbf{Hazard pay:} All subcontracted workers must receive a retroactive addition of \$5/hour in hazard pay to their base hourly rate in recognition of the increased risk they have faced and continue to face in performing their regular duties during the pandemic.
    
\end{itemize}

\vspace{-0.5em}
\subsection*{Safe and Transparent Working Environments.}
\addcontentsline{toc}{subsection}{Safe and Transparent Working Environments.}%
%\item A safe and transparent working environment:
\begin{itemize}
    %\renewcommand{%\labelitemi}{--}
    \item
    \textbf{Clear communication of COVID-related risks:} Subcontracted workers are placed in potentially dangerous working situations without being provided with adequate knowledge about the risks they face. For example, workers may be required to clean the room of someone who has tested positive for COVID-19 without being informed that this is the case. Staff must receive prompt and accurate communication, in a language and medium that is accessible to them, of the precise nature of any COVID-related risk to them in each of their workplace environments, including, but not limited to, the number of confirmed COVID-19 cases on campus, their locations of origin, and their relative proximity to each employee’s working environment.
    
    
    \item
    \textbf{Fair, manageable workloads:} In compensating for the drastic reduction of the size of the workforce through widespread layoffs, Stanford has forced workers to take on increasingly demanding workloads. These inhumane cost-cutting measures not only threaten to degrade the quality of the work performed by subcontracted laborers, but also pose a grave risk to the health of our entire community. Stanford must implement practices that ensure the fairness and feasibility of workloads before any layoffs may occur, including, but not limited to, workplace walkthroughs and clear demonstrations of workload reduction. 
    
    \item
    \textbf{Reinstatement of laid-off workers:} As mentioned, increased sanitation workloads due to current health needs place an unmanageable burden on current employees. Allowing previously laid-off workers to return would both meet the pressing safety needs of the entire campus community and reduce the strain on  current employees. 
    
    \item
    \textbf{Guarantees of no retaliation:} When confronted with unsafe working conditions, such as the lack of adequate PPE provision, subcontracted workers can neither refuse to work, nor can they demand improvements to their working conditions, for fear of retaliation (e.g. having their hours cut, losing shifts, being laid off entirely). This requires workers to choose between risking their health or their livelihood. Stanford must commit to a proactive anti-retaliation program to protect workers who refuse to work within unsafe conditions. 
    
    \end{itemize}
%\end{enumerate}



%\subsection{Just compensation for essential work}
% \subsection*{Just compensation for essential work}
% \addcontentsline{toc}{subsection}{Just compensation for essential work}%



%\subsection{Safe and transparent working environments}
% \subsection*{Safe and transparent working environments}
% \addcontentsline{toc}{subsection}{Safe and transparent working environments}%




\newpage

\section*{Protect Survivors of Sexual Violence.}
\addcontentsline{toc}{section}{Protect Survivors of Sexual Violence.}%

\vspace{.5em} % here bc otherwise there's a negative space? 
\hline\hline

\subsection*{Create Exceptions from COVID-19 Disciplinary Rules for \\Victims of Sexual Violence.}
\addcontentsline{toc}{subsection}{Create Exceptions from COVID-19 Disciplinary Rules for Victims of Sexual Violence.}%

As part of the University’s response to the COVID-19 pandemic, students who are physically on campus are required to sign the Stanford Campus Compact. While the Compact exacerbates  many of the problems addressed in this Roadmap, the Compact will also have harmful consequences for survivors of sexual violence or Intimate Partner Violence (IPV). 

Under the Compact, students are not allowed to have guests, and are prohibited from being indoors without a face covering with people who are not members of their household. Since IPV and sexual violence often happen under such circumstances, survivors will be deterred from reporting sexual violence if they have violated the Compact prior to or during their assault. This will prevent the University from learning about and addressing sexual violence on campus, making it easier for perpetrators to re-offend and making the entire community less safe. Sexual violence and IPV are already under-reported at Stanford (see the \href{https://provost.stanford.edu/wp-content/uploads/sites/4/2019/10/AAU-2019-Survey-Stanford-University-Report-and-Appendices.pdf}{2019 AAU Campus Climate Survey}); the University must make a serious effort to encourage rather than discourage reporting of sexual violence.

This harmful impact will be particularly pronounced for already marginalized members of the Stanford community who have the most to lose if removed from Stanford: international students fearing termination of their visas, low-income students, and those with unstable home environments who fear eviction from Stanford housing.

Stanford should follow the steps of some of our peer institutions (such as \href{https://www.brown.edu/about/administration/title-ix/Title\%20IX\%20grievance\%20Procedure}{Brown}) that have granted students amnesty from COVID-19 rules for instances of sexual violence. While the University is \href{https://files.covid19.ca.gov/pdf/guidance-higher-education--en.pdf}{mandated by the state and county} to create guidance for students regarding social distancing, mask-wearing, and other forms of risk reduction, the University enjoys wide discretion in deciding which violations of the Compact to pursue in a disciplinary process. Accordingly, the University should not initiate disciplinary processes against victims of sexual violence (or IPV) for violations of the Compact tied to their assault. 

Sexual violence and IPV are serious public health risks of their own. It is in the best interest of the entire campus community to encourage reporting of sexual violence. This can only be achieved by declining to pursue disciplinary charges against victims of sexual violence or IPV who violated COVID-19 rules.

\subsection*{Mitigate the Harm of Betsy DeVos’s Title IX Policies.}
\addcontentsline{toc}{subsection}{Mitigate the Harm of Betsy DeVos’s Title IX Policies.}%

Since the Department of Education (hereinafter “Department”) released updated guidelines regarding the enforcement of Title IX in May, both ASSU and Sexual Violence Free Stanford have called on the University to implement these regulations in a way that considers the safety of survivors past, present, and future first and foremost. In the months since then, Stanford has failed to do this. Instead, the extensive feedback of student leaders and nearly 1,500 community members has largely been ignored. The Department’s new regulations present a dangerous threat to student safety, and the Stanford administration has failed to seize the numerous, legally viable opportunities to defuse the damage of these regulations as other institutions across the country have done. SVFS, in coordination with a student committee that has been meeting weekly with administration, \href{https://docs.google.com/document/d/1a6Had9yfTEgL6VGv1V8YzJJ7aedvQAryt-YRbKbRZqI/edit}{submitted 25 \textit{legally viable} requests for the implementation of the new Title IX policy} informed by the limitations and opportunities of the new regulations. The vast majority of these were disregarded, referencing legal infeasibility without providing any explanation or adequate justification beyond that.

\textbf{Substantive discussions regarding the revision of the Title IX policy should be conducted transparently, respectfully, and continuously with Sexual Violence Free Stanford, survivor advocates, and the Stanford community at large until the policy is deemed final at an appropriate time, with clearly outlined opportunities for revision granted until that point.}

The necessary revisions to the University’s implementation approach are as follows:

\begin{enumerate}[label=\textbf{\arabic*.}]
    \item \textbf{Expand legal aid for all students going through the Title IX process.} While the administration’s new policy extends the legal aid hours to students in cases that continue to a hearing to 11 preparation hours, this concession will only benefit the vast minority of students whose cases actually extend to a hearing. Most students’ cases proceed to an informal resolution, and this number will likely only increase due to the extreme cross-examination procedures of the new regulations. Stanford’s new policy continues to leave these students unsupported and in fact \textbf{cuts the number of legal aid hours from 9 to 6}. Furthermore, the legal aid with which students are provided continues to be restricted to lawyers hand-selected by the Stanford administration who have historically been mistrusted by survivors. The administration must expand the amount of legal aid provided to students in both informal resolution processes and hearings in order to adequately meet the need of a Title IX proceeding. The University must allow this aid to be applied to the attorney of a party’s choice, inclusive of both cases falling within the jurisdiction of Title IX and also those falling under ancillary processes for sexual misconduct in development by the University. This extension of time as well as attorney option are necessary to rectify the systemically unjust nature of sexual violence proceedings against survivors of this trauma. This policy is not unprecedented; according to \href{https://sexualmisconduct.princeton.edu/policy}{Princeton’s new Title IX policy}, students may receive legal funding for their personally selected attorney based on need. 
    % \vspace{1em}

    Just legal aid is not only an accommodation Stanford must provide students in order to demonstrate equity and support after experiencing institutional trauma; it is also about creating a safer campus that prevents assault and misconduct from continuing to occur. 90\% of sexual assaults on campus are committed by repeat offenders. Until Stanford commits to a Title IX proceeding that effectively generates accountability so that previous offenders do not continue their behavior, our campus will not be safe from sexual violence. The University must create a safe environment for all students, while directing funds away from violent approaches such as SUDPS towards these more effective systemic solutions.

    \item \textbf{Establish a Title IX proceeding timeline that is more reflective of the urgency and trauma inherent in this process.} The University denied SVSF’s request that Stanford retain a 60-day timeline for Title IX proceedings as mandated by the 2011 regulations and has provided no acceptable justification for its decision to implement a 120-day timeline with an additional 30 days granted for an appeal process. Stanford’s position on this matter means that a survivor’s Title IX proceedings---which are consistently re-traumatizing experiences---could drag on for five months or longer. Peer institutions \href{https://titleix.harvard.edu/files/titleix/files/harvard_university_interim_title_ix_sexual_harassment_student_procedures.pdf}{Harvard} and \href{https://sexualmisconduct.princeton.edu/policy}{Princeton} have both committed to 90-day timelines; there is no reason that Stanford cannot do the same. The administration needs to consider how extra time for them to do their jobs comes at the direct expense of students’ emotional and academic success. 

    \item \textbf{Expand hearing protections to mitigate the harms of cross-examination.} Questions during the cross-examination period should be written and submitted in advance for approval in order to reduce badgering and harassing questions. The University has previously dismissed this request, stating that written questions submitted in advance violate the Department’s requirement that cross-examination be conducted “live” and “orally.” However, advanced written screening is an essential and non-mutually exclusive protection from the harm of a court-style cross-examination process. \href{https://www.brown.edu/about/administration/title-ix/Title\%20IX\%20grievance\%20Procedure}{Brown utilizes a similar iteration} of this type of protection in its hearing policy, which requires that questions be submitted in writing to the hearing officer at least two days in advance of the hearing so that the officer is able to approve or reject questions on the basis of relevancy and/or decorum guidelines. 

    \item \textbf{Limit opportunity for University interference in the Title IX process.} The draft proposal of Stanford’s Title IX guidelines creates multiple opportunities for interference by the University that have historically proven to break survivor trust. Multiple clauses throughout the Title IX policy allow intervention or overriding by the Title IX coordinator, a position currently held by Cathy Glaze despite long-standing student mistrust of Ms. Glaze and her culpability in \href{https://provost.stanford.edu/wp-content/uploads/sites/4/2018/04/OCR-Redacted-Stanford-Letter-of-Finding.pdf}{Stanford’s 2015 Office of Civil Rights Investigation}. Ms. Glaze has failed to effectively and equitably perform the responsibilities of Title IX Coordinator. Though not required by federal regulations, current Stanford policy grants authority to the Title IX Coordinator to conduct informal resolutions, determine mandatory dismissals of cases, and to continuously override the determinations of the Hearing Officer throughout a proceeding; \textit{the University must revoke this nearly unbounded authority.}  

\end{enumerate}

% \newpage
\subsection*{Fund Inclusive Measures to Support Survivors in their Path to Healing and Justice.}
\addcontentsline{toc}{subsection}{Fund Inclusive Measures to Support Survivors in their Path to Healing and Justice.}%

Stanford must engage with impacted students and advocates to fund intentional and inclusive outlets for justice and support in the aftermath of trauma. Specialized Confidential Support Team (CST) counselors trained in assisting and identifying with LGBTQ+ and Black students should replace campus police in crisis response. \href{https://vaden.stanford.edu/cst/who-we-are}{No existing CST counselors} meet this extraordinary need.

Stanford must also take substantive action toward the integration of restorative and transformative justice in the sexual violence disciplinary process. A commitment should be made to fund restorative justice mechanisms by hiring community-trained specialists and actively promoting well-researched restorative justice options in the informal resolution process. 

\subsection*{Stop Betraying Survivors and Student Advocates.}
\addcontentsline{toc}{subsection}{Stop Betraying Survivors and Student Advocates.}%

Stanford must respect the efforts of survivors and student advocates and follow through on its promises. In order to create a campus that supports and cares for survivors, Stanford should respect survivors’ voices and genuinely address their concerns, especially when directly soliciting them for input. Specifically, Stanford must take concrete steps and commit to:

\begin{itemize}[itemsep=1em]
    \item \textbf{Responding to student input from the August $4^{\mathrm{th}}$-$9^{\mathrm{th}}$ comment period.} Stanford administrators promised to host an open student comment period during which the student body would be able to share input during the drafting process. Over 120 individual and group comments were submitted under the assumption that this was an opportunity for meaningful input, and over 1,300 students signed an SVFS petition on the topic. However, Stanford administrators never meaningfully and substantively engaged with the extensive feedback that was shared by students – and much of this input was summarily dismissed. The only response to the feedback were two short memos posted on the Institutional Equity and Access website without any announcement or email to students. We were led to believe the comment period would resemble a discussion, in which students and administrators could iteratively listen and respond to each others’ perspectives. What we received was a black box into which students could send their feedback without any substantive explanation as to why suggestions were rejected, and no further opportunity to influence the policies. Furthermore, Stanford administrators refused to meet with survivors and student advocates for the remainder of August, providing zero opportunities for clarification regarding the new policy. 
    \vspace{1em}
    
    Stanford has an institutional trust problem when it comes to sexual violence. The events of the past month make it abundantly clear why this distrust is so pervasive across the student body. If Stanford administrators wish to begin to repair this relationship, they should start by taking meaningful steps to engage with student feedback on the Title IX and ancillary processes. 

    \item \textbf{Transparency regarding why, when, and how the Title IX process will be changed in the future.} Since the new Title IX process went into effect on August 14th, it has been revised multiple times without notice to students. However, students who devote significant time and labor to provide input on these policies deserve to know when – and why – the Title IX process gets revised. Furthermore, even though administrators promised to continue revising the Title IX process in response to student feedback throughout the fall, they have not shared any details regarding how student feedback will be solicited and how it will be transparently addressed. In order to make up for the lack of engagement with student feedback during the initial comment period, it is vital that Stanford follow through on this promise and establish a clear process to collect and respond to student input in the future. This process should also include a transparent system to document and explain policy changes, so that all students can understand when and why the Title IX process is revised.

    \item \textbf{Creating a process to help survivors of online and off-campus sexual violence.} The new Title IX regulations are limited in scope to acts of sexual violence committed on campus or at university programs and within the US. In response to student advocacy, Stanford administrators committed to establishing an ancillary process for all offenses no longer covered under Title IX. To date, no ancillary process has been released. By contrast, several peer institutions---Harvard, Yale, Princeton, Columbia, and the UC system---have all published information on their ancillary processes alongside the new Title IX policy. Stanford must publish their ancillary process immediately and substantively engage with student feedback in the process.
    
\end{itemize}

\newpage
\section*{Meet Students' Basic Needs.}
\addcontentsline{toc}{section}{Meet Students' Basic Needs.}%
\vspace{0.5 em}\hline\hline

\textit{We envision a Stanford where the basic needs of all students are met. 
}

Students’ basic needs have not been met, particularly during the ongoing pandemic. When Stanford chose to send home undergraduate students in March, many students were left without finances for travel, short term housing, or other incidental emergency costs that occurred due to the sudden change in plans. The Community Offerings spreadsheets created by members of the Basic Needs Coalition generated hundreds of requests as students struggled to meet their basic needs during a time of crisis. This struggle continues today; many students are unable to work campus jobs, families and partners have lost income, and additional costs for necessities like PPE and mental healthcare are piling up. The Stanford Basic Needs Coalition started a community based \href{https://basic-needs-stanford.org/}{Basic Needs Fund} to help students with necessities like food, shelter and healthcare. As of September $7^{\mathrm{th}}$ there have been 149 requests for basic needs support, totaling over \$120,000. While Stanford has some resources in place to help certain students, many of these resources have exclusionary restrictions. Additionally, many of these resources are decentralized and not well-advertised, leaving students unaware of the resources Stanford can provide. 

The problem of students’ basic needs going unmet is not specific to the pandemic. Even in more stable times, a \href{https://regents.universityofcalifornia.edu/regmeet/july16/e1attach.pdf}{University of California study} found that 40\% of undergraduates and 25\% of graduate students experience low or very low food security. Peer institutions such as Duke and Cornell found similar food security problems, and it is all but certain that conditions are similar at Stanford. A recent survey of the postdoctoral population found that nearly 10\% of postdocs at Stanford suffer from food insecurity. Many Stanford students struggle to pay for the cost of dependent healthcare and childcare, and high housing costs lead some students to opt-out of permanent housing altogether. Further, as the University of California study found, low food security correlates with poor mental health outcomes and lower grades. The struggle for basic needs is a huge barrier to a just and equitable education, and these conditions disproportionately affect FLI students, students of color, non-traditional students, and other already marginalized groups. 

\subsection*{Short-Term Considerations.}
\addcontentsline{toc}{subsection}{Short-Term Considerations.}%

\textit{COVID-19 should not place undue financial stress on students or otherwise prevent basic needs from being met.}

As Stanford reopens, we know that plans will be in flux. Students have already faced the  uncertain status of campus jobs, housing, and financial support. We believe Stanford should do all it can to avoid having the burden of these changes fall on vulnerable students, and have transparent plans for any decisions they make. Stanford should:

\begin{itemize}
    \item Release a transparent plan for providing funds for housing and travel to students who are told to leave campus.
    \newline
    \item Commit to pay continuance for any workers (including student workers and subcontracted workers) who must self-quarantine or leave campus.
    \item Allow graduate students to use Emergency grant-in-aid money for food, housing, and other basic needs for the duration of the pandemic, and preferably beyond.
    \item Increase both funding and development efforts for the FLI Office Opportunity Fund, and make emergency grants-in-aid available to all students regardless of expected tuition contribution for the duration of the pandemic.
\end{itemize}

Changes in finances are not the only way students’ basic needs are affected by the pandemic response. In order to ensure the holistic welfare and safety of students, Stanford should: 

\begin{itemize}
    \item Release a transparent plan to ensure students’ access to high-quality food if dining halls have to close or restrict hours.
    \item Make masks and other PPE free of charge at campus housing centers.
    \item Offer full CAPS services to students who defer enrollment or take a leave of absence during the pandemic.
    \item Guarantee CAPS and Vaden access for students regardless of whether they have signed the Housing Compact, or future COVID-related agreements.
    \item Guarantee students who defer or take a leave of absence during the pandemic are covered by quality healthcare insurance. 

\end{itemize}

\subsection*{Long-Term Vision.}
\addcontentsline{toc}{subsection}{Long-Term Vision.}%

\textit{The basic needs of all Stanford students are met. }

Here is what we envision:

\begin{itemize}
    \item Stanford ensures housing, healthcare, and childcare affordability for all students, as outlined earlier in the Roadmap.
    \item Stanford continually assesses whether the basic needs of marginalized groups (such as BIPOC and LGBTQ+ students) are being met.
    \item Through the Office of Institutional Research, Stanford implements a campus-wide basic needs survey modeled after the Housing and Food Survey Instruments designed by the UC Office of the President.
    \item Stanford anticipates environmental pressures on food security, housing, and other basic needs and responds with directed and meaningful support. This could include a campus food pantry, rent relief, and emergency supplies for health and environmental crises. 
    \item Stanford creates a dedicated physical Basic Needs Center that can direct and assist students in accessing Stanford resources as well as government resources (MediCal, CalFresh, etc) when necessary. Stanford is responsive to the specific needs of students of color, students on visas, students who are undocumented, students who identify as LGBTQ+, and other groups when creating this office.

\end{itemize}


\newpage 

\vspace*{5.5em}
\begin{center}
\large\textbf{Follow the Reverse Town Hall Coalition on Social Media}
\end{center}
\vspace*{1em}
Abolish Stanford: \href{https://twitter.com/abolishstanford}{Twitter} $\vert$ \href{https://www.instagram.com/abolishstanford/}{Instagram} $\vert$ \href{https://lists.riseup.net/www/subscribe/abolishstanford/}{Email List}

Black Graduate Student Association: \href{https://www.facebook.com/StanfordBGSA}{Facebook} $\vert$ \href{https://www.instagram.com/stanfordbgsa/}{Instagram} $\vert$ \href{https://twitter.com/StanfordBGSA}{Twitter}

Graduate Student Council: \href{https://www.facebook.com/gsc.stanford/}{Facebook} $\vert$ \href{https://twitter.com/StanfordGSC}{Twitter}

Sexual Violence Free Stanford: \href{https://www.instagram.com/svfreestanford/}{Instagram}

Stanford Basic Needs Coalition: \href{https://www.facebook.com/basicNeedsAtStanford}{Facebook} $\vert$ \href{https://www.instagram.com/basicneedsatstanford/}{Instagram}

Stanford Neighbor Accountability Coalition

Stanford Solidarity Network: \href{https://www.facebook.com/stanfordsolidaritynetwork}{Facebook} $\vert$ \href{https://twitter.com/solidarityntwrk}{Twitter}

Students for Workers’ Rights: \href{https://www.facebook.com/StanfordSWR}{Facebook} $\vert$ \href{https://twitter.com/stanford_swr}{Twitter} $\vert$ \href{https://www.instagram.com/stanfordswr/}{Instagram}

\vspace*{1.5em}
\begin{center}
\href{https://docs.google.com/forms/d/e/1FAIpQLSfSbbNOngWJa3JpLRSug4KqUhRjWr414RARM9sXy5XbCNuAcg/viewform}{\Large\textbf{Reach Out to Us to Get Involved}}
\end{center}


\end{document}




